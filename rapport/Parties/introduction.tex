\chapter*{Introduction}
\addcontentsline{toc}{chapter}{Introduction}

%Préambule de l'introduction

Lors d'un projet universitaire organisé dans notre Master Informatique à l'Université de Lille 1, nous avons eu l'occasion de travailler sur un projet de recherche basé sur la simulation multi-agents, proposé par deux enseignants-chercheurs en Informatique de l'Université de Lille 1 et de l'Université de Rennes.
\\
Ceci a eu pour but d'approfondir notre connaissance dans le milieu de la Recherche, ainsi que nous confronter à la simulation multi-agents.

%Contexte de l'introduction
\section*{Contexte}

Le projet a pour but la création d'un \textit{serious-game} permettant à l'utilisateur de simuler un écosystème marin complexe sur lequel agit les conditions propres à celui-ci, mais aussi le comportement des agents implémentés. Le contexte du \textit{serious-game} permettra ainsi de pouvoir informer et mettre en garde l'utilisateur sur les répercutions engendrées par l'activité de la pêche sur cet écosystème, comme par exemple les étudiants de l'Université de Rennes.

%Problèmatique de l'introduction
\section*{Problèmatique}

La problématique du projet s'axe sur:
\begin{itemize}
\item{la difficulté de la modélisation du système multi-agents et des ressources halieutiques,}
\item{l'implémentation de cette modélisation,}
\item{l'expérimentation et le paramètrage de l'ensemble de la modélisation.}
\end{itemize}

%Objectif de l'introduction
\section*{Objectif}

Pour répondre à la problématique, nous avons utilisé la plateforme de simulation Netlogo et l'extension développée dans l'équipe SMAC de CRIStAL : IODA.